% 华东师范大学博士(硕士)论文主格式,请修改前三行以保证格式符合要求。

%==============格式配置开始==============%
\def \degree {phd} % phd, master
\def \year {2025}
\def \draftfigure {off} % on,off
\def \docstyle {normal} % normal,tmlc,anonymous
\def \refstyle {auto} % auto,manual

% student id
\def \stuID {52xxxxxx}
% title
\def\thesisTitle{论文标题}
\def\thesisTitleNoWrap{论文标题}
\def\thesisETitle{
    Title
}
%==============格式配置结束==============%

\def \dc {on}
\ifx \draftfigure \dc
\documentclass[12pt,a4paper,fancyhdr,openany,twoside,draft]{ctexbook}
\else
\documentclass[12pt,a4paper,fancyhdr,openany,twoside]{ctexbook}
\fi

\def \phd {phd}
\ifx \degree \phd
    \def\degreeCN{博士}
    \def\degreeENs{Doctoral}
\else
    \def\degreeCN{硕士}
    \def\degreeENs{Master's}
\fi

\input{./format/format.tex}

\def \dd {anonymous}
\def \de {tmlc}
\ifx \docstyle \dd
    \notanonymousfalse % 开启该行则为盲审版本,注释该行则为非盲审版本(即查重或终稿版本)
\else
    \ifx \docstyle \de
        \tmlctrue % 开启该行则为查重版本,注释该行则为终稿版本(仅在上一行在注释状态下有效)
    \fi
\fi

% 以下是正文部分

\usepackage{emptypage}
\usepackage{tikz}
\usepackage{bbm}
%	\usepackage{algorithm}
\usepackage{multirow}
\usepackage{bbm}
% \usepackage{algpseudocode}
\usepackage{wrapfig}
\usepackage{graphicx}
\usepackage{subfigure}
\usepackage{array} % Necessary for new column definitions
\usepackage{listings}
% Define \tabincell
\newcommand\tabincell[2]{\begin{tabular}{@{}#1@{}}#2\end{tabular}}
%\usepackage{algorithm,algpseudocode}
\usepackage{lipsum}
%\usepackage{algorithmic}
%\usepackage{subcaption}
\usepackage{amsfonts}
\usepackage{xcolor}
\newtheorem{Proposition}{Theorem}
\DeclareMathOperator*{\argmax}{arg\,max}
\DeclareMathOperator*{\argmin}{arg\,min}
\newtheorem{proposition}{Proposition}
\theoremstyle{definition}
\newtheorem{definition}{Definition}
\newtheorem{assumption}{Assumption}
\theoremstyle{remark}
\usepackage{enumitem}
\usepackage{pifont}
\usepackage{xspace}
\newcommand{\greencheck}{{\checkmark}}
\newcommand{\xmark}{\ding{55}}%
% \newcommand{\redx}{{\color{red}\xmark}}
\newcommand{\redx}{{\xmark}}
% \usepackage{algorithmic}

\usepackage{xeCJK} % 声明包
%\renewcommand{\algorithmcfname}{算法}
\floatname{algorithm}{算法}
\renewcommand{\algorithmicrequire}{\textbf{输入:}}
\renewcommand{\algorithmicensure}{\textbf{输出:}}
\begin{document}
\input{preface/inner-cover.tex}
\cleardoublepage
\newpage

\pagestyle{empty}

\noindent{\large Dissertation for {\degreeENs} Degree in {\year}}\\
\hspace*{\fill} {\large University Code: 10269}\\
\hspace*{\fill} {\large Student ID: \ifnotanonymous \stuID \else *********** \fi}

\vskip 2cm

\begin{center}
    {\Huge $\mathbb{EAST}\,\mathbb{CHINA}\,\mathbb{NORMAL}\,
            \mathbb{UNIVERSITY}$}
\end{center}

\vskip 3cm

\begin{center}
    \bfseries{\scshape{\huge {\thesisETitle}}} \\
\end{center}

\vskip 2cm

{\large
\begin{center}
\begin{tabular}{r}
    Department:         \\
    % \\
    Major:              \\
    Research Direction: \\
    Supervisor:         \\ 
    Candidate:
\end{tabular}
\begin{tabular}c
    % 盲审需要注释下面一些信息,注意页面上的学号是否需要注释

    School of xxx \\
    \hline xxx \\
    \hline  xxx  \\
    \hline \ifnotanonymous A/Prof. xxx \ \ Prof. xxx \else *** \fi \\ %Prof. Xiaoling Wang  \\
    \hline \ifnotanonymous xxx \else *** \fi \\
    % ~~~~~~~~~~~~~~~~~~~~~~~~~~~~~~~~~        \\
    % \hline ~~~~~~~~~~~~~~~~~~~~~~~~~~~~~~~~~ \\
    % \hline ~~~~~~~~~~~~~~~~~~~~~~~~~~~~~~~~~ \\
    % \hline ~~~~~~~~~~~~~~~~~~~~~~~~~~~~~~~~~ \\
    % \hline ~~~~~~~~~~~~~~~~~~~~~~~~~~~~~~~~~ \\

    \hline
\end{tabular}
\end{center}
}


\vskip 5cm  

\begin{center}
    {\Large November, 2024}
\end{center}

\cleardoublepage

\ifnotanonymous %非盲审,也就是查重版,不显示原创性+名单;终版全要
    \iftmlc
    \else
        \input{preface/copyright.tex} 
        \cleardoublepage
        %\newpage
\pagestyle{empty}
$$\\ \\ \\ $$

\centerline{\bf\Large ${\mbox{\kaishu { }}}\,\,${\degreeCN}学位论文答辩委员会成员名单}


\vskip 10mm

\begin{center}
{
\renewcommand{\arraystretch}{1.75}
\large
\begin{tabular}{| m{25mm}<{\centering}| m{25mm}<{\centering}| m{45mm}<{\centering}| m{25mm}<{\centering}|}\hline 
 {\heiti 姓名}&{\heiti 职称} & {\heiti 单位} &{\heiti 备注}  \\\hline
 
  & & & {\heiti 主席}\\\hline
  & & & {\heiti }\\\hline
 & & & {\heiti }\\\hline
 & & & {\heiti }\\\hline
  & & & {\heiti }\\\hline

\end{tabular}
}
\end{center}

       % \cleardoublepage
    \fi
\else %盲审仅显示原创性,不要名单
    \input{preface/copyright.tex} %查重不显示
    \cleardoublepage %查重不显示
\fi

\setlength{\baselineskip}{25pt}  %正文设为25磅行间距
\newpage
\pagenumbering{roman}
\pagestyle{plain}

\chapter*{\xiaosan\heiti{摘\quad 要}}
\addcontentsline{toc}{chapter}{摘要}
摘要内容

摘要内容

摘要内容

\vspace{0.5cm}
% \hspace{-1cm}
\sihao{\heiti{关键词:}}\xiaosi{关键词1,关键词2,关键词3,关键词4,关键词5}


\cleardoublepage
\newpage
\vspace{-1cm}
\chapter*{\zihao{-3}\heiti{ABSTRACT}}
\addcontentsline{toc}{chapter}{Abstract}
\vspace{-0.5cm}

Abstract content

Abstract content

Abstract content

\vspace{0.5cm}
\hspace{-1cm}
{\bf{\sihao{Keywords:}}} \textit{keywords1, keywords2, keywords3, keywords4, keywords5}






































\cleardoublepage

\input{format/graphic.tex}


\chapter{绪\quad 论}


\section{研究背景与意义}



\section{挑战}


\section{本文主要工作}

\section{本文组织结构}


\cleardoublepage
\chapter{相关工作综述}
XXX
\section{A方法}

\subsection{a}
 
\subsection{b}

\subsection{c}

\section{B方法}

\subsection{a}
 
\subsection{b}

\subsection{c}

\section{C方法}

\subsection{a}
 
\subsection{b}

\subsection{c}
\cleardoublepage
\chapter{工作一}

\section{算法介绍}

\subsection{一个好的训练过程}
\begin{algorithm}[!htp]
  \caption{训练过程}\label{alg_train}
  \begin{algorithmic}[1]
  \REQUIRE 特征 $a$,\par 
           矩阵 $adj$,\par 
           矩阵 $adj\_n$                  %输入条件
  \ENSURE 输出 $l\_adj$     %输出
  \STATE \textbf{begin}
      \STATE {好}
      \STATE {$l\_a$ $\leftarrow$ 映射函数($a$,$adj\_n$)}
      \STATE {好}
      \STATE {$l\_adj$ $\leftarrow$ 映射函数($adj$,$l\_a$)}
  
  \RETURN {$l\_adj$}
  \STATE \textbf{end}
  \end{algorithmic}
  \end{algorithm}

\cleardoublepage
\chapter{工作二}

\section{算法介绍}






\subsection{一个好的调度算法}
\begin{algorithm}[htb!]
\caption{xxx调度算法}
\label{alg:schem}
\begin{algorithmic}[1]
\STATE \textbf{初始化:}$Q$-网络参数:$\theta$,目标$Q$-网络参数:$\theta^{-}$,最佳状态-动作-r缓冲区$\mathcal{B}$,折扣因子$\gamma$,最大周期$K$;
\FOR{${\hbox{Episode}} = 1, \cdots, K$}
\STATE 从数据集中采样一个请求$\text{Seqs}$并将$s_0$发送给调度器;
    \FOR{$t=0,\cdots,T$且$d_t \neq False$}
        \STATE 使用基于(\ref{eq: sampling})的XXX获得XXX$a_t$;
        \STATE XXXX$r_t, s_{t+1}, d_{t+1}$;
        \STATE 将XXX$s_t, a_t, r_t, s_{t+1},  d_{t+1}$并按照(\ref{eq:episodic})更新XXX$\mathcal{B}$;
        \STATE XXX(\ref{eq:loss})更新$Q$-网络;
        \STATE 根据(\ref{eq:tar})软更新目标$Q$-网络。
    \ENDFOR
\ENDFOR
\end{algorithmic}
\end{algorithm}


\cleardoublepage
\chapter{工作三}

\section{一个等式}

此处引用等式\ref{eq:1}和等式\eqref{eq:1}。

\begin{equation}\label{eq:1}
    \begin{array}{ll}
    x_{1} & =y_{1} \\
    x_{2} & =\left(y_{2}\right)
    \end{array}
\end{equation}
\cleardoublepage
\chapter{工作四}
\cleardoublepage
\chapter{总结与展望}

\section{研究工作总结}

\section{研究工作展望}

\cleardoublepage

% 两种文献导入方式。如果开启bst自动模式,在format.cls中bibcontrol部分无效
\ifx \refstyle \auto
    % bst自动
    \addcontentsline{toc}{chapter}{参考文献}
    \bibliographystyle{references/gbt7714-2005}
    \bibliography{references/paper.bib}
    
\else
    % bib手动
    \begin{thebibliography}{999}  
    \addcontentsline{toc}{chapter}{参考文献}
    \input{references/paper-manual.bib}
    \end{thebibliography}
\fi

\cleardoublepage

%\appendix
%\input{chapters/appendix.tex}
%\cleardoublepage

\ifnotanonymous %非盲审,查重:不要致谢,终稿要
    \iftmlc
    \else
        \addcontentsline{toc}{chapter}{致谢} %查重不显示
        \chapter*{致\qquad 谢\markboth{致\qquad 谢}{}}

完整致谢。

\vspace{0.8cm} \hspace{10cm}  学位申请人姓名

\hspace{8cm}  二零二四年十一月于普陀校区


  %查重不显示
        \cleardoublepage %查重不显示
    \fi
\else %盲审,需要短的致谢
    \addcontentsline{toc}{chapter}{致谢}
    \input{acknowledgement/acknowledgement-tmlc.tex}
    \cleardoublepage
\fi

\addcontentsline{toc}{chapter}{发表论文和科研情况}
\chapter*{\centering{\songti{攻读{\degreeCN}学位期间科研情况}}}

% \vskip 5mm
%\noindent{\heiti $\blacksquare$ 作者简历}\vskip 5mm
%XXX
\noindent{\heiti $\blacksquare$ 学术论文}\vskip 5mm
\noindent{\heiti $\blacksquare$ 以第一作者身份完成的学术论文}\vskip 5mm

[1] \ifnotanonymous \textbf{真实名字}, 导师真实名字, 其余作者, et al.  \else \textbf{本文作者}, 本文作者导师, 其余作者. \fi Title. \textbf{(CCF-X,对应第X章内容)}     

[2] \ifnotanonymous \textbf{真实名字}, 导师真实名字, 其余作者, et al.  \else \textbf{本文作者}, 本文作者导师, 其余作者. \fi Title. \textbf{(SCI,IF:0.1,对应第X章内容)}     

[3]  \ifnotanonymous \textbf{真实名字}, 导师真实名字, 其余作者, et al.  \else \textbf{本文作者}, 本文作者导师, 其余作者. \fi Title. \textbf{(CCF-X,对应第X章内容)}   


\noindent{\heiti $\blacksquare$ 以参与者身份发表的学术论文}\vskip 5mm
\iffalse
[1] Tong P, \textbf{Zhang Q}, Yao J. Leveraging domain context for question answering over knowledge graph[J]. Data Science and Engineering, 2019, 4(4): 323-335. (CCF-C,ESCI,EI,IF:5.1)

[2] Yang Y, \textbf{Zhang Q}, Yao J. Task-Driven Neural Natural Language Interface to Database[C]//International Conference on Web Information Systems Engineering. 2023: 659-673. (CCF-C)

[3] Li, Q; \textbf{Zhang, Q}; Yao J, et al. Event extraction for criminal legal text, 11th IEEE International Conference on Knowledge Graph, ICKG 2020.     (EI)

[4] Pu, T; \textbf{Zhang, Q}; Yao J, et al. Medical entity extraction from health insurance documents, 11th IEEE International Conference on Knowledge Graph, ICKG 2020.     (EI)
\fi

[1] \ifnotanonymous 真实名字, \textbf{真实名字}, 真实名字.  \else 第一作者, \textbf{本文作者}, 本文作者导师. \fi Title. (CCF-X,SCI,EI,IF:0.2)  

[2] \ifnotanonymous 真实名字, \textbf{真实名字}, 真实名字.  \else 第一作者, \textbf{本文作者}, 本文作者导师. \fi Title. (CCF-X,SCI,EI,IF:0.2)  

[3] \ifnotanonymous 真实名字, \textbf{真实名字}, 真实名字.  \else 第一作者, \textbf{本文作者}, 本文作者导师. \fi Title. (CCF-X,SCI,EI,IF:0.2) 


\bigskip\bigskip


\ifnotanonymous \noindent{\heiti $\blacksquare$ 参与基金项目}\vskip 5mm 

\begin{itemize}
  \item 国家自然科学基金xxxx项目,XXXX,20xx-20xx。
  \item XXXX, Institute of Digital Life, CAS.
\end{itemize}

\else \fi




\bigskip\bigskip\bigskip



\cleardoublepage
\printindex
\end{document}
